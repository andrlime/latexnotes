\section{Lecture 10}
``I want to go to a lower level''
\subsection{The Definition of Convergence}
Let $a_1, a_2, ...$ be some sequence of real numbers, and let $a \in \mathbb{R}$. We say that the sequence $a_n \to a$ if
\begin{definition}
    A sequence $a_1, a_2, ...$ converges if for every positive number $\varepsilon > 0$, $\exists N \in \mathbb{N}$ such that for every $n > N$, $\abs{a_n - a} < \varepsilon$.
\end{definition}
\noindent Some real analysist's remarks:
\begin{enumerate}
    \item This is still mid-level math. What is a sequence? What is a real number? What is minus?
    \item This is a complicated condition. Prove $\forall \varepsilon$ and $\forall N$
\end{enumerate}

\begin{lemma}
    The sequence $a_n := 1/n$ converges to $0$.
\end{lemma}
\begin{proof}
    Given $\varepsilon > 0$, take $N = \ceil{1/\varepsilon} + 1$, we show that the condition in the definition of convergence holds
    \begin{align}
        \abs{a_n - a} = \abs{1/n - 0} = 1/n < 1/N < \varepsilon
    \end{align}
\end{proof}

\subsection{Back to Fourier Analysis}
\begin{definition}
    Let $f_n$ be a sequence of functions, $f \in \mathcal{F}$. We say
    \begin{enumerate}
        \item $f_n \to f$ \textbf{pointwise} if for every $x_0$, $f_n(x_0) \to f(x_0)$, i.e., $\forall x_0$ and $\forall \varepsilon$, $\exists N$ such that $\forall n > N$
        \begin{align}
            \abs{
                f_n(x_0) - f(x_0)} \le \varepsilon
        \end{align}
        \item $f_n \to f$ \textbf{uniformly} if $\forall \varepsilon$, $\exists N$ such that $\forall x_0$ and $\forall n > N$,
        \begin{align}
            \abs{f_n(x_0) - f(x_0)} < \varepsilon
        \end{align}
    \end{enumerate}
\end{definition}
\begin{lemma}
    $f_n(x) = x^n$ on $[0,1]$, $f_n(x) \to g(x)$ pointwise where
    \begin{align}
        g(x) = \begin{cases}
            0&0 \le x < 1\\
            1&x=1
        \end{cases}
    \end{align}
    but $f_n(x) \not\to g(x)$ uniformly.
\end{lemma}
\begin{proof}
    For every $n$, $x^n$ is continuous. So, by the Intermediate Value Theorem, there is some point $a_n < 1$ such that $f_n(a_n) = 1/2$. $\forall n$,
    \begin{align}
        \abs{f_n(a_n) - g(a_n)} = \frac{1}{2} \not< \left[ \varepsilon = \frac{1}{2} \right]
    \end{align}
    which means for $\varepsilon = \frac{1}{2}$, the definition does not hold.
\end{proof}
