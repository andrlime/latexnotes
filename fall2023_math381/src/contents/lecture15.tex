\section{Lecture 15}
\subsection{More on Continuity}
\begin{lemma}
    $f: \mathbb{L} \to \mathbb{C}$ is continuously differentiable $\implies$ $n \cdot a_n(f) \to 0$
\end{lemma}
\begin{proof}
    Recall
    \begin{align}
        a_n(f') = -2\pi in \cdot a_n(f)
    \end{align}
    Since $f$ is continuously differentiable, $f'$ is continuous, so
    \begin{align}
        a_n(f') \to 0
    \end{align}
\end{proof}
In the other direction\footnote{In class, this was actually written mirrored on the board.}
\begin{lemma}
    \begin{align}
        \begin{cases}
            \sum_{n \in \mathbb{Z}} \underbrace{\abs{n a_n(f)}}_{b_n} < \infty\\
            f\text{ continuous}
        \end{cases} \implies f \text{ is continously differentiable}
    \end{align}
\end{lemma}
\begin{proof}
    From Lecture 14 (implied?)
    \begin{align}
        \sum -2\pi in \cdot a_n(f) \cdot e^{2\pi in x} < \infty
    \end{align}
    converges (pointwise and uniformly) to a continuous function $g$. Then, let
    \begin{align}
        h(x) = \int g(x) \dd{x}
    \end{align}
    Then,
    \begin{align}
        h'(x) = g(x)
    \end{align}
    so,
    \begin{align}
        a_n(g) = -2\pi in \cdot a_n(h) &\implies a_n(h) = \frac{a_n(g)}{-2\pi in} = \frac{-2\pi in \cdot a_n(f)}{-2\pi in}\\
        &\implies a_n(h) = a_n(f)
    \end{align}
    i.e. the Fourier coefficients are the same.
\end{proof}
\begin{lemma}
    It remains that we need to prove that for functions $h, f \in \mathcal{F}$
    \begin{align}
        a_n(h) = a_n(f) \implies h = f
    \end{align}
\end{lemma}
\begin{proof}
    \begin{align}
        \norm{h-f}^2 &= \int_0^1 \left(h(x) - f(x)\right)^2 \dd{x}\\
        &= \sum_n \abs{a_n(h-f)}^2
    \end{align}
    But,
    \begin{align}
        (15.11) = \sum_n \abs{a_n(h)-a_n(f)}^2 = 0
    \end{align}
    Furthermore, if for any $x_0$,
    \begin{align}
        h(x_0) \ne f(x_0)
    \end{align}
    then we would get
    \begin{align}
        h(x) \ne f(x)
    \end{align}
    over some interval $I \subset [0,1]$, but then
    \begin{align}
        &\int_I \abs{h(x) - f(x)}^2 \dd{x} > 0\\
        &\implies \int_0^1 \abs{h(x) - f(x)}^2 \dd{x} \ge \int_I \abs{h(x) - f(x)}^2 \dd{x} > 0
    \end{align}
    This contradicts (15.12), so for all $x_0$,
    \begin{align}
        h(x_0) = f(x_0)
    \end{align}
    which implies
    \begin{align}
        h - f = 0
    \end{align}
\end{proof}
\subsection{Summary}
\begin{lemma}
    If $f$ is $k$-ce continuously differentiable, then $n^k \cdot a_n(f) \to 0$
\end{lemma}
\begin{lemma}
    If
    \begin{align}
        \sum \abs{a_n(f) \cdot n^k} \to 0
    \end{align}
    then $f$ is $k$-ce continuously differentiable.
\end{lemma}
\subsection{A Corollary}
\begin{lemma}
    $f$ is $\infty$-ce differentiable $\iff$ for every $k$, $n^k \cdot a_n(f) \to 0$
\end{lemma}
\noindent Now it's an if and only if, so let's prove it.
\begin{proof}
    We saw $\Rightarrow$ in Lemmas 16.1-3. What about $\Leftarrow$? Assume the right hand side. Then, we want to show that
    \begin{align}
        f \text{ is $p$ times differentiable}
    \end{align}
    Then, for $k = p + 2$,
    \begin{align}
        n^{p+2} \cdot a_n(f) \to 0
    \end{align}
    and
    \begin{align}
        \abs{n^{p+2} \cdot a_n(f)} < M
    \end{align}
    for some $M$. But,
    \begin{align}
        \sum \abs{n^p \cdot a_n(f)} &= \sum \abs{n^{p+2} \cdot a_n(f)} \frac{1}{n^2}\\
        &\le \sum M \frac{1}{n^2} < \infty
    \end{align}
    And by Lemma 16.5 this proves that $f$ has $p$ derivatives. This can be applied for all $p \in \mathbb{Z}$, and if we replace all instances of $p$ with $k$, this proves the Lemma.\footnote{This proof is bad induction.}
\end{proof}

\subsection{Heat Equation Again}
Let $h(x)$ be a continuous function and let $u(t, x)$ be a solution of the heat equation with initial condition $u(0, x) = h(x)$. This solution is
\begin{align}
    u(t, x) = \sum c_n \cdot e^{-4\pi^2n^2t} \cdot e^{2\pi in x}
\end{align}
Because $h(x)$ is continuous, $c_n \to 0$ and are bounded by some constant $\max_{x\in[0,1]}\:h(x)$. So,
\begin{align}
    u(0.000001, x) = \sum c_n \cdot e^{-4\pi^2n^2(0.000001)} \cdot e^{2\pi in x}
\end{align}
This decays faster than any polynomial\footnote{This was not proven}. So by Lemma 26, this implies
\begin{align}
    u(0.000001, x) \text{ is $\infty$-ce differentiable}
\end{align}

\subsection{Alcoholism and Convolutions}
Let
\begin{align}
    g(t):= \text{ BAC $t$ minutes after taking a shot}
\end{align}
Suppose you drink $a_0$ shots at $t=0$, $a_1$ shots at $t=1$, $\cdots$, $a_n$ shots at $t=n$. In general, the total BAC at time $t$ is
\begin{align}
    a_0\cdot g(t) + a_1\cdot g(t-1) + \cdots
\end{align}
These are such important questions that they lead to two definitions
\begin{definition}
    The convolution of two sequences $a_n$ and $b_n$ ($n \in \mathbb{Z}$) is the sequence
    \begin{align}
        (a * b)_n := \sum_{k \in \mathbb{Z}} a_k b_{n-k}
    \end{align}
\end{definition}
\noindent And, for periodic functions,
\begin{definition}
    The convolution of two periodic functions $f$ and $g$ is
    \begin{align}
        (f * g)(x) := \int_0^1 f(t)\cdot g(x-t) \dd{t}
    \end{align}
\end{definition}
\noindent We claim
\begin{theorem}
    \begin{align}
        a_n(f * g) = a_n(f) \cdot a_n(g)
    \end{align}
\end{theorem}
\begin{proof}
    \begin{align}
        a_n(f * g) &= \int_0^1 (f*g)(x) e^{-2\pi in x}\\
        &= \int_0^1 \int_0^1 f(t) \cdot g(x-t) \cdot e^{-2\pi in t} \cdot e^{-2\pi in (x-t)} \dd{t} \dd{x}\\
        &= \int_0^1 \int_0^1 f(t) \cdot g(x-t) \cdot e^{-2\pi in t} \cdot e^{-2\pi in (x-t)} \dd{x} \dd{t}\\
        &= \int_0^1 f(t) \cdot e^{-2\pi in t} \cdot \int_0^1 g(x-t)  \cdot e^{-2\pi in (x-t)} \dd{x} \dd{t}
    \end{align}
    Fix $y = x-t$,
    \begin{align}
        a_n(f * g) &= \int_0^1 f(t) \cdot \cdot e^{-2\pi in t} \dd{t} \cdot \int_{-t}^{1-t} g(y)  \cdot e^{-2\pi in y} \dd{y} \\
        &= \int_0^1 f(t) \cdot e^{-2\pi in t} \dd{t} \cdot \int_{0}^{1} g(y) \cdot e^{-2\pi in y} \dd{y}\\
        &= a_n(f) \cdot a_n(g)
    \end{align}
\end{proof}